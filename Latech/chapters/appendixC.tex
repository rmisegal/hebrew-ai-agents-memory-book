\hebrewsection{נספח ג: \en{\texttt{gmail-extractor.md}}}

\textbf{תיאור הסוכן ויכולותיו:}

שרת \en{MCP} למיצוי אימיילים מ\en{-Gmail} על בסיס תוויות וטווחי תאריכים, עם ייצוא לפורמט \en{CSV} ותמיכה מלאה ב\en{-Unicode}.

\textbf{הגדרות שרת \en{MCP}:}
\begin{itemize}
\item \textbf{שם שרת:} \en{gmail-extractor}
\item \textbf{פרוטוקול:} \en{stdio}
\item \textbf{פקודת הפעלה:} \en{python3 /path/to/gmail\_mcp\_server.py}
\end{itemize}

\textbf{פרמטרים לפונקציה \en{search\_and\_export\_emails}:}
\begin{itemize}
\item \en{label}: תווית \en{Gmail} לסינון (אופציונלי)
\item \en{start\_date}: תאריך התחלה בפורמט \en{YYYY-MM-DD}
\item \en{end\_date}: תאריך סיום בפורמט \en{YYYY-MM-DD}
\item \en{max\_results}: מקסימום תוצאות (ברירת מחדל: \num{100})
\end{itemize}

\textbf{דוגמה לתגובת \en{JSON}:}

\begin{english}
\begin{verbatim}
{
  "success": true,
  "count": 15,
  "message": "Successfully exported 15 emails",
  "output_file": "csv/Research_Data_emails.csv"
}
\end{verbatim}
\end{english}
