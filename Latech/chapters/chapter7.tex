\hebrewsection{האמנזיה של המכונה: הזיכרון כבסיס לציוויליזציה הדיגיטלית}

\hebrewsubsection{הרקע ההיסטורי-פילוסופי: מכתב יתדות למרחב קונטקסט}

מאז ומעולם, הקפיצה הקוגניטיבית הגדולה ביותר של האנושות לא נבעה משיפור הזיכרון הביולוגי עצמו, אלא מהיכולת להנדס "זיכרון חוץ-גופי". המצאת הכתב, הפיכת סיפורים לארכיונים ממלכתיים וחקיקת חוקות על גבי לוחות אבן, יצרו את הבסיס לציוויליזציה על ידי אחסון ידע מחוץ למוחו של אדם יחיד. כלי הזיכרון החיצוניים הללו אפשרו יצירת פרויקטים ארוכי-טווח, שחייבו קוהרנטיות ורציפות ידע לאורך דורות וזמנים.

מודלי שפה גדולים (\en{LLMs}), ובפרט מודל הקידוד של \en{Anthropic}, \en{Claude Code}, מתמודדים כיום עם אותה מגבלה יסודית שדרשה מהאנושות להמציא את הארכיון: מגבלת אחסון וצריכת אנרגיה קוגניטיבית. למרות כוחם החישובי יוצא הדופן, מודלים אלה הם "חסרי מצב" (\en{Stateless}) מטבעם, וסובלים מ"אמנזיה קונטקסטואלית" בין סשנים. חלון הקונטקסט, שהוא המקבילה החישובית ל"זיכרון העבודה" הביולוגי, הוא משאב יקר ומוגבל. כאשר חלון זה מתמלא, או כאשר המשתמש מנקה אותו בכוונה כדי לשפר את תוצאות המודל, כל הקונטקסט הקודם נעלם.

בפרויקטים מורכבים וארוכי-אופק, הטרגדיה הזו מתבטאת בשכפול משימות, חוסר עקביות ארכיטקטונית וצורך מתמיד להסביר מחדש לקלוד את מהות הפרויקט והכללים הפנימיים שלו. קיים צורך דוחק לא רק בזיכרון קונטקסטואלי מובנה (\en{Code memory}) אלא גם במנגנונים לאחזור ידע ספציפי ועדכני (כדוגמת \en{Retrieval-Augmented Generation – RAG}). הפתרון ההנדסי לבעיה זו, אשר הופיע מתוך קהילת המפתחים והפך לפרקטיקה דומיננטית, הוא יצירת מערכת \en{Claude Code memory} המבוססת על ארכיון חיצוני מובנה.

\hebrewsubsection{הגדרת \en{Claude Code memory}}

מערכת \en{Claude Code memory}, כפי שהוגדרה בפרקטיקות הקהילתיות המתקדמות, היא ארכיטקטורה מבוססת קבצי \en{Markdown}, המיועדת להנדס "זיכרון עבודה חיצוני, קריא ופרסיסטנטי" בתוך ספריית הפרויקט. ארכיטקטורה זו מורכבת מארבעה עמודי ליבה, שכל אחד מהם ממלא תפקיד קוגניטיבי מוגדר בניהול הפרויקט:

\begin{enumerate}
  \item \textbf{\en{PRD.MD} (\en{Product Requirements Document}):} המגדיר את \textbf{מה} בונים.
  \item \textbf{\en{CLAUDE.MD}:} המגדיר את \textbf{איך} עובדים – ספר החוקים הקנוני.
  \item \textbf{\en{PLANNING.MD}:} המפרט את האסטרטגיה הטכנולוגית והארכיטקטורה.
  \item \textbf{\en{TASKS.MD}:} המנהל את הביצוע בפועל ואת מעקב ההתקדמות.
\end{enumerate}

הקמת מערכת זו אינה בגדר "טריק" תכנותי אלא שכפול מודרני של מבנים ארגוניים קדומים, הנדרשים ליציבות ארוכת-טווח. ניתן לראות כיצד המערכת משכפלת את מבנה הניהול הממלכתי: ה\en{-PRD} הוא החוקה (המטרות העסקיות), ה\en{-CLAUDE.MD} הם דיני העבודה הפנימיים (הקאנון הארגוני), ה\en{-PLANNING.MD} הוא תוכנית החומש (האסטרטגיה), וה\en{-TASKS.MD} הוא יומן השינויים המבצעי (\en{Ledger}). מבנה זה מכריח את הסוכן לפעול באופן שיטתי וממושמע, בדומה למהנדס אנושי בעל דיסציפלינה.

בפרקים הבאים (פרקים \num{8}–\num{13}) נצלול לעקרונות ההנדסיים, נבחן את ההבחנה בין זיכרון מובנה לבין אחזור מידע דינמי, ונציג את הפרקטיקות המתקדמות להבטחת עקביות ארכיטקטונית לאורך זמן. המעבר מ"סוכן רגעי" ל"שותף קוגניטיבי" מתחיל כאן, בהבנת האמנזיה הבסיסית של המכונה ובפיתרון המבני הפשוט והמהפכני כאחד – הארכיון החיצוני.
