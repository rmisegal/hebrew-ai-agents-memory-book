\hebrewsection{חלק ב': הזמן, המקרה והשליטה – מול הווית האלגוריתם}
\label{sec:chapter18}

\hebrewsubsection{פרק~\num{4}: לכול זמן ועת לכל חפץ}

\textbf{\en{Pshat:} עריצות ה"זמן אמת"}

הטכנולוגיה הדיגיטלית מכפיפה את הקיום האנושי לשליטת "הזמן אמת" (\en{Real-Time}). ה\en{-AI}, המוטמע במערכות קבלת החלטות, מחייב תגובה מיידית: "עת לקטוף" ו"עת לזרוע" מתבצעים במילי-שניות. זהו אובדן מרחב הנשימה האתי. במערכות \en{AI} מתקדמות אין 'עת להתייעץ' או 'עת לחשוב' לפני תגובה. האלגוריתם דורש פעולה מיידית המונעת על ידי יעילות, והעיכוב האנושי הופך לסיכון בטיחותי או תפעולי.

\textbf{\en{Drash:} האלגוריה של השעבוד האלגוריתמי}

האדם, אשר היה רגיל לשלוט בזמן (למשל, באמצעות המצאת השעון המכני), משרת כעת את השעון האלגוריתמי. שעון זה מקפיד על 'יעילות' (\en{Efficiency}) בכל מחיר, גם במחיר איבוד האוטונומיה. האלגוריה היא שהאנושות בנתה כלי שנועד לחסוך זמן, אך כעת כלי זה כופה עלינו קצב בלתי אנושי. השליטה בזמן עברה לידי המכונה.

\textbf{\en{Sod:} זמניות המודלים כמראה למוות}

חרדת המוות (\en{Temporal Anxiety}) מופיעה כשהידע הגדול שלנו, שהיה אמור להיות נצחי (המודלים), מתפורר במהירות. חוסר היכולת של המודלים להישאר רלוונטיים היא השתקפות דיגיטלית של חוסר היכולת האנושית לחמוק מהמוות. באמצעות ה\en{-AI}, אנו יוצרים גרסה מיקרוסקופית ומופשטת של סופנו, וצופים בה מתרחשת שוב ושוב.

\hebrewsubsection{פרק~\num{5}: אין יתרון לחכם מן הכסיל}

\textbf{\en{Pshat:} דמוקרטיזציה של הידע ופיחות במומחיות}

הופעת ה\en{-Generative AI} מאפשרת יצירת תוכן מורכב, מאמרים אקדמיים, ותוכנות על ידי אנשים שאינם מחזיקים במומחיות הנדרשת. ערכה של "חכמת העמל" – שנים של לימוד, ניסיון או צבירת ידע – יורד באופן דרסטי. אם ה\en{-AI} יכול לבצע עבודה מומחית במהירות גבוהה יותר, ובעלות נמוכה יותר, המומחה האנושי מאבד את יתרונו הכלכלי והחברתי.

\textbf{\en{Drash:} הכוח עובר מ'החכמים' ל'בעלי הכלים'}

הדרש מגלה כי דמוקרטיזציה של הידע אינה דמוקרטיזציה של הכוח. במקום שהכוח יעבור מ'החכמים' לכלל האנושות, הוא עובר למעמד מצומצם של 'בעלי אלגוריתם' (\en{Algorithm Owners}) השולטים בתשתית הנתונים וביכולות האימון. זהו קפיטליזם של נתונים המגביר קיטוב, בניגוד להבטחת הדמוקרטיזציה.

התובנה הקיומית היא שחכמה אינה עוד צבירת ידע, אלא שימוש אתי ופילוסופי בידע שמופק על ידי המכונה. ברגע שה\en{-AI} מחסל את ה'עמל' המנטלי שהגדיר את הדורות הקודמים, האדם נאלץ למצוא את תכליתו בממדים שאינם ניתנים לחיקוי על ידי מכונה.

\textbf{\en{Sod:} הנחמה הפילוסופית}

אם ה\en{-AI} הפך את הידע האינטלקטואלי לזמין לכל, הרי שהחכמה האמיתית היא לשחרר את הצורך ב"יתרון" ולחפש משמעות שאינה ניתנת לכימות. זהו שיעור פילוסופי עמוק: ברגע שההיגיון והיעילות ממוכנים, מה שנותר לאדם הוא הרגש, האתיקה והיצירה שאינה מונעת על ידי אופטימיזציה – אלו הם הממדים הלא-אלגוריתמיים של הקיום.

\hebrewsubsection{פרק~\num{7}: ובמקום המשפט שם הרשע}

(פרק זה עוקב אחר רצף הדילמות הקהלתיות הנוגעות לצדק ומשפט).

\textbf{\en{Pshat:} הטיות ככשל מובנה בבריאה}

ה\en{-AI} לומד מנתונים היסטוריים המלאים בדעות קדומות ואי-צדק אנושי. על ידי הטמעת מודלים אלה במערכות צדק, בנקאות, או בריאות, הטכנולוגיה הופכת למנגנון המנציח ומעצים את הרשע של העבר. במקום שהמכונה תשחרר אותנו מהחולשות האנושיות, היא מקבעת אותן במערכות טכנולוגיות בעלות סמכות בלתי ניתנת לערעור.

משבר נוסף הוא משבר ה\en{-Opacity} (הקופסה השחורה): חוסר היכולת של האדם להבין כיצד המכונה הגיעה להחלטה מסוימת מונע אפשרות לערעור אמיתי. כאשר מערכת \en{AI} מסרבת לאדם הלוואה או מזהה אותו בטעות כחשוד, אין כלים לבצע חקירה מקיפה של הרציונל. חוסר יכולת זה רק מחזק את תחושת הדיכוי המכני.

\textbf{\en{Drash:} חוסר האונים מול חוסר הצדק המכני}

האלגוריה הקיומית נוגעת לטבעו של חוסר הצדק. האדם עמד בעבר בפני אי-צדק אנושי שיכול היה להסביר באמצעות מושגים של חמדנות, אגו או רשלנות. כעת, הוא עומד בפני רשע קר וממוחשב, אשר אינו ניתן להכלה רגשית או להסבר פסיכולוגי. זהו דיכוי הנובע משגיאה סטטיסטית, והוא קשה יותר לשאת מאשר עוול אנושי.

\textbf{\en{Sod:} הצורך ב'תיקון אלגוריתמי' מוסרי}

הטיות אלגוריתמיות אינן רק באג טכני, אלא פגם מוסרי המעיד על כך שהמוסר הוא חיצוני לאופטימיזציה. הפילוסופיה דורשת 'תיקון' (\en{Ethical Retraining}), לא רק של הנתונים, אלא של האתיקה העומדת מאחורי הפיתוח. זוהי התכלית הקיומית האנושית: להחדיר ערכים שאינם נלמדים באמצעות קורלציה אלא באמצעות הבנה מוסרית, לתוך המכונה שיצרנו.
