\hebrewsection{ארכיטקטורת תודעה דיגיטלית: בניית סוכן \en{MCP} עבור \en{Gmail}}

לאחר שגיבשנו את הרקע הפילוסופי וההיסטורי, נעבור מן המופשט אל המוחשי. פרק זה מספק תוכנית פעולה לבניית סוכן \textbf{AI} מעשי ומתמחה – שרת \en{MCP} עבור \en{Gmail}. אין זה תרגיל תיאורטי גרידא, אלא מדריך שלב-אחר-שלב לבניית יחידת בסיס פונקציונלית במערכת רב-סוכנים. במהלכו נתקן תפיסות שגויות שפורסמו בעבר, ונספק מתודולוגיה מאובטחת ויעילה.

\hebrewsubsection{מהמיתוס למציאות: התפתחות \en{MCP SDK}}

ראשית, ראוי להבהיר עובדה היסטורית חשובה: דיווחים מוקדמים התייחסו ל\en{-Google MCP Server ADK} (ערכת פיתוח סוכן) זמינה לשימוש. \textbf{בתחילת \hebyear{2025}, לא קיימה ערכה כזו}. הרצון בפתרון פלא "מהמדף" היה מובן, אך מפתחים נאלצו לבנות את הרכיבים הללו בעצמם מאפס.

\textbf{עם זאת, המצב השתנה}: קהילת ה\en{-Model Context Protocol} פרסמה \en{MCP Python SDK} רשמי – ספרייה המפשטת משמעותית את בניית שרתי \en{MCP}. כעת קיימים \textbf{שני מסלולים} לבניית סוכן \en{Gmail}:

\begin{enumerate}
\item \textbf{גישה ידנית (ללא \en{SDK}):} בניית שרת \en{MCP} מאפס עם טיפול ידני בפרוטוקול, ניתוב בקשות, וסריאליזציה של נתונים
\item \textbf{גישה עם \en{SDK}:} שימוש ב\en{-MCP Python SDK} הרשמי (חבילת \en{mcp} ב\en{-PyPI}) המספק תשתית מוכנה, דקורטורים לכלים, וניהול אוטומטי של הפרוטוקול
\end{enumerate}

\textbf{דרישות גרסה:} הגישה עם \en{SDK} (נספח ה) דורשת \en{Python 3.10} ומעלה. הגישה הידנית (נספחים א–ד) תומכת בגרסאות \en{Python} מוקדמות יותר.

בפרק זה נציג \textbf{את שני המסלולים}. הגישה הידנית (נספחים א-ד) מלמדת את יסודות הפרוטוקול ומעניקה שליטה מלאה. הגישה עם \en{SDK} (נספחים ה-ו) מציעה פיתוח מהיר יותר ותחזוקה קלה יותר. בחירת הגישה תלויה בצרכי הפרויקט: למערכות ייצור מורכבות, ה\en{-SDK} מומלץ; ללמידה והבנה עמוקה, הגישה הידנית בעלת ערך.

\hebrewsubsection{השילוש הקדוש של אימות: הבטחת אבטחת הסוכן}

כדי שסוכן יפעל בעולם האמיתי ויגש לנתונים אישיים, עליו להתבסס על יסודות אבטחה מוצקים. הסוכן שלנו דורש "שילוש" של אישורים ואמצעי אימות:
\begin{enumerate}
  \item \textbf{הרשאות גישה ל\en{-Gmail}:} יש להגדיר פרויקט ב\en{-Google Cloud} ולאפשר את \en{Gmail API}. הדבר כולל קבלת מזהה \en{Client ID} וסוד \en{Client Secret}, והשלמת תהליך \en{OAuth 2.0} לקבלת אסימון גישה לחשבון ה\en{-Gmail} של המשתמש.
  \item \textbf{מפתחות \en{API} לפלטפורמת ה\en{-AI}:} לשילוב הסוכן במערכות \en{AI} חיצוניות כגון \en{Claude CLI}, נדרש מפתח \en{API} תקף (למשל, מפתח שירות מאנתרופיק עבור \en{Claude} או מפתח מודל \en{Gemini} של גוגל). יש לשמור מפתחות אלה באופן מאובטח (בקובץ \en{\texttt{.env}} מקומי) כדי למנוע דליפה.
  \item \textbf{בקרת סביבה והרשאות מערכת:} הסוכן רץ כתהליך מקומי, ולכן יש להקפיד על הגבלת ההרשאות שלו. למשל, להפעילו כמשתמש רגיל ללא הרשאות מנהל מערכת, ולהגבילו לתיקיות ונתונים הנחוצים בלבד. תקשורת ה\en{-MCP} בין הסוכן לבין \en{Claude CLI} נעשית בערוץ סטנדרטי (\en{STDIO}) מוגן, כך שאין גישה לא מבוקרת לסביבת הסוכן.
\end{enumerate}

מצוידים באמצעי האימות הללו, ניגש למלאכת הבנייה עצמה. ראשית, נכין סביבת פיתוח פייתון עם הספריות הדרושות (ראו \en{\texttt{requirements.txt}} בנספח ד). נפתח שרת \en{MCP} ייעודי בשפת \en{Python} שמתחבר ל\en{-Gmail API}, מחפש הודעות לפי קריטריונים, ומייצא תוצאות לקובץ \en{CSV} בפורמט \textbf{Unicode} תקני. במהלך הפיתוח נדגיש התייחסות נכונה לתווי עברית ולכיווניות (למשל, נוודא הוספת \en{BOM} לקובצי \en{CSV} כדי להבטיח קריאות תקינה בתוכנות כ\en{-Excel}).

לאחר כתיבת קוד הליבה של הסוכן (ראו נספח א לקוד המלא ונספח ב לדוגמת שימוש), נערוך בדיקות יסודיות. למשל, נריץ חיפוש לדוגמה על תיבת \en{INBOX} בטווח תאריכים מוגבל כדי לוודא שהסוכן מאתר מספר הודעות הצפוי ומייצא קובץ תקין. נוודא שתוכן בעברית אינו נפגם (כלומר, שלא מתקבל ג'יבריש או "????" במקום טקסט קריא). בהצלחה, צפויה תגובת \en{JSON} מהסוכן עם \en{\texttt{"success": true}}, מונה ההודעות שמצא, והמסלול לקובץ ה\en{-CSV} שנוצר.

בנקודה זו בנינו רכיב בסיס עצמאי: סוכן \en{MCP} פעיל עבור \en{Gmail}. כעת ניצב בפנינו אתגר השילוב – לצרף את הסוכן הבודד למערך סוכנים נרחב יותר באמצעות פלטפורמת \en{Claude CLI}, ובכך לממש אורקסטרציה חכמה של משימות מורכבות.

\hebrewsubsection{השוואה טכנית: יישום ידני מול \en{MCP Python SDK}}

לאחר שהצגנו את הגישה הידנית, ננתח כעת את ההבדלים המרכזיים בין שני מסלולי היישום. השוואה זו תסייע בבחירה מושכלת בין הגישות.

\textbf{יתרונות הגישה הידנית (נספח א):}
\begin{itemize}
\item \textbf{שליטה מלאה:} גישה ישירה לכל היבט של הפרוטוקול וניהול השרת
\item \textbf{למידה עמוקה:} הבנת מנגנוני \en{MCP} ברמה הנמוכה ביותר
\item \textbf{התאמה אישית:} יכולת לשנות כל חלק בהתאם לצרכים ספציפיים
\item \textbf{ללא תלות חיצונית:} אין תלות בספריית צד שלישי שעלולה להשתנות
\end{itemize}

\textbf{חסרונות הגישה הידנית:}
\begin{itemize}
\item \textbf{זמן פיתוח ארוך:} צורך בכתיבת קוד תשתית נרחב (ניתוב, סריאליזציה, טיפול בשגיאות)
\item \textbf{תחזוקה מורכבת:} כל שינוי בפרוטוקול דורש עדכון ידני
\item \textbf{סיכון לשגיאות:} יישום עצמאי של פרוטוקול מורכב מגדיל סיכוי לבאגים
\item \textbf{חוסר סטנדרטיזציה:} קוד שונה מפרויקט לפרויקט, קושי בשיתוף פעולה
\end{itemize}

\textbf{יתרונות \en{MCP Python SDK} (נספח ה):}
\begin{itemize}
\item \textbf{פיתוח מהיר:} דקורטור \en{@tool} פשוט הופך פונקציה לכלי \en{MCP} זמין
\item \textbf{קוד תמציתי:} הקוד קצר פי \num{2}-\num{3} לעומת הגישה הידנית
\item \textbf{תחזוקה קלה:} ה\en{-SDK} מטפל אוטומטית בשינויים בפרוטוקול
\item \textbf{תיעוד אוטומטי:} \en{docstrings} של הפונקציות הופכים לתיאור הכלי ב\en{-MCP}
\item \textbf{בדיקות מובנות:} ה\en{-SDK} כולל כלי בדיקה ואימות מובנים
\end{itemize}

\textbf{חסרונות \en{MCP Python SDK}:}
\begin{itemize}
\item \textbf{תלות חיצונית:} שינויים ב\en{-SDK} עשויים לשבור קוד קיים
\item \textbf{הסתרת מורכבות:} קושי בדיבוג בעיות ברמת הפרוטוקול
\item \textbf{גמישות מוגבלת:} קשה ליישם דפוסים לא סטנדרטיים
\end{itemize}

\textbf{דוגמה קונקרטית – הגדרת כלי:}

בגישה הידנית, הגדרת כלי דורשת:
\begin{itemize}
\item יצירת מילון \en{JSON} מפורט עם שם, תיאור, ופרמטרים
\item כתיבת פונקציית \en{handler} שמנתבת קריאות לפונקציה הנכונה
\item טיפול ידני בסריאליזציה של קלט ופלט
\item ניהול מצב השרת ואימות פרמטרים
\end{itemize}

עם \en{MCP Python SDK}, אותו כלי מוגדר בשורה אחת:
\begin{english}
\begin{verbatim}
@tool(name="search_emails", description="Search Gmail")
async def search_emails(label: str, start_date: str):
    ...
\end{verbatim}
\end{english}

ה\en{-SDK} מייצר אוטומטית את מפרט ה\en{-JSON}, מטפל בניתוב, ומבצע אימות טיפוסים.

\textbf{המלצות לבחירה:}
\begin{itemize}
\item \textbf{למידה אקדמית / הבנת יסודות:} התחילו עם הגישה הידנית (נספח א)
\item \textbf{אבות-טיפוס מהירים / פרויקטי סטארט-אפ:} השתמשו ב\en{-SDK} (נספח ה)
\item \textbf{מערכות ייצור קריטיות:} שקלו גישה היברידית – פיתוח עם \en{SDK}, הבנה עם הגישה הידנית
\item \textbf{צוותים גדולים:} ה\en{-SDK} מספק סטנדרטיזציה ומפחית עקומת למידה
\end{itemize}

לסיכום, \textbf{שתי הגישות תקפות}. הגישה הידנית מעניקה שליטה והבנה; ה\en{-SDK} מעניק מהירות ונוחות. בפרקטיקה, מומלץ להכיר את שתיהן: להבין את המנגנון הפנימי דרך הגישה הידנית, ולהשתמש ב\en{-SDK} בפיתוח יום-יומי.
