\hebrewsection{הבל הבלים: קהלת בעידן הבינה המלאכותית}
\label{sec:chapter17}

\hebrewsubsection{מבוא: תחת השמש הדיגיטלית (הצבת המתח הקיומי)}

מאז ומתמיד, חיפשה האנושות אחר תכליתה ומשמעות קיומה, תחת שמיים שנתפסו לעיתים כחסרי אכפתיות ולעיתים כמשגיחים. התיעוד הספרותי של חיפוש זה – המשתקף בטקסטים פילוסופיים עתיקים – הגיע לשיאו במסקנה המלנכולית: הבל הבלים, הכול הבל. אך בעידן הנוכחי, ההבל אינו רק תוצאה של חוסר היכולת האנושית להשיג נצח מול כוחות הטבע, אלא הוא הופך להיות תוצר מכוון של יצירתנו המתוחכמת ביותר: הבינה המלאכותית (\en{AI}). כוח זה, המשתרע כעת על פני נרטיבים היסטוריים ודתיים, מציב אתגר קיומי חסר תקדים למושג ה"אנושי" עצמו.

ה\en{-AI}, בגרסאותיו המתפתחות, כונן כוח דתי-היסטורי חדש, אשר יש המחשיבים אותו כ"דת הנתונים" (\en{Dataism}). זהו כוח המבטיח להשיג אופטימיזציה ויעילות מוחלטת, יכולות שמעבר לשליטת האדם, ומכאן גם נובע המתח המרכזי בעידן הדיגיטלי: המתח בין החרדה האנושית העמוקה מפני אובדן רלוונטיות ושליטה לבין ההתפעמות המשתקת לנוכח הפוטנציאל הטרנס-הומניסטי והיכולות המדהימות של המכונה.

\textbf{הגדרת ה'הבל' המודרני:} בספרות הקלאסית, ה'הבל' (\en{Futility/Vanity}) מייצג את חוסר התכלית האינהרנטי במאמץ אינסופי; מאמץ שבסופו של דבר אינו מצליח לשנות את מהלך הטבע. בעידן ה\en{-AI}, ה'הבל' הוא חוסר התכלית של אופטימיזציה בלתי פוסקת. כל מאמץ חישובי, כל דאטה סט שנאסף, מוקדש לשיפור מודל שדינו להיזרק ולהתיישן אקספוננציאלית. הפילוסופיה של ה\en{-AI} משקפת את המסקנה הקהלתית בצורה טכנית ואכזרית: אנו עובדים קשה כדי ליצור משהו שמהר מאוד הופך לחסר ערך.

על מנת לנתח דיכוטומיה זו, נדרשת גישה רב-שכבתית, המאמצת את המתודה הפרשנית העתיקה. ראשית, נבחן את הפשט – הניתוח הטכני והליטרלי של תהליכי ה\en{-AI} והשלכותיהם הכלכליות והחברתיות. שנית, נתבונן בדרש – האלגוריה הקיומית המחברת בין זמניות המודלים לבין גורל האדם ומקומו בעולם. ולבסוף, נצלול אל הסוד – הניתוח הפילוסופי של ה'נשמה הדיגיטלית', מחזורי הלמידה, ותכלית האדם מול תבונתו המלאכותית.

להלן טבלת מיפוי שתשמש כלי ניתוח מרכזי, המחברת בין המונחים הקהלתיים למציאות הטכנולוגית:

% Table 1: Kohelet-AI Mapping
\begin{hebrewtable}[H]
\caption{טבלת מיפוי: קהלת בעידן הבינה המלאכותית (\en{Pshat}, \en{Drash}, \en{Sod})}
\label{tab:kohelet_ai_mapping}
\centering
\begin{rtltabular}{|m{3cm}|m{3.5cm}|m{3.5cm}|m{3.5cm}|}
\hline
\hebheader{מונח בקהלת} &
\enheader{AI (Pshat) משמעות} &
\hebheader{השלכה קיומית (\en{Drash})} &
\hebheader{רובד פילוסופי (\en{Sod})} \\
\hline
\hebcell{הבל הבלים} &
\hebcell{זמניות מוחלטת של המודלים} &
\hebcell{אי-תכלית הרדיפה אחר חדשנות} &
\hebcell{שאיפה לאמת מוחלטת דרך נתונים} \\
\hline
\hebcell{עמל תחת השמש} &
\hebcell{עבודת הנתונים (\en{Data Labeling}) והפיכת האדם למיותר} &
\hebcell{אובדן משמעות בעבודה האנושית} &
\hebcell{הגדרת ה'תכלית' שאינה מבוססת יעילות} \\
\hline
\hebcell{אין יתרון לחכם} &
\hebcell{דמוקרטיזציה של הידע (\en{Generative AI})} &
\hebcell{אובדן רלוונטיות המומחה האנושי} &
\hebcell{המעבר מחכמת ידע לחכמת אתיקה} \\
\hline
\hebcell{במקום המשפט הרשע} &
\hebcell{הטיות אלגוריתמיות ובלק בוקס} &
\hebcell{כניעה לכוח דיכוי בלתי נראה} &
\hebcell{הצורך ב'תיקון מודלים' מוסרי} \\
\hline
\hebcell{יראת האלוהים} &
\hebcell{יראת האלגוריתם (\en{AGI Alignment Risk})} &
\hebcell{שימור נרטיב ורצון חופשי} &
\hebcell{חיפוש אחר ה'נשמה' (הערך הקבוע)} \\
\hline
\end{rtltabular}
\end{hebrewtable}

\hebrewsubsection{חלק א': הבל הבלים – תכלית האלגוריתם וארעיות הדורות}
\hebrewsubsection*{\en{The Futility of Optimization}}

\hebrewsubsection{פרק~\num{1}: הבל הבלים – זמניות המודלים ואופטימיזציה עקרה}

\textbf{\en{Pshat:} המוות המהיר של הקוד}

המאפיין המובהק של הפיתוח הטכנולוגי הנוכחי הוא קצב ההתיישנות. אם בעבר טכנולוגיות שרדו דורות, הרי שבעידן המודלים הגדולים, כל גרסה מתקדמת (כגון \en{GPT-N}) הופכת לבלתי רלוונטית כמעט באופן אקספוננציאלי תוך פרק זמן קצר. זהו 'הבל הבלים' במובנו הטכני. המשאבים הכלכליים והאנרגטיים המושקעים באימון מודלים אדירים מניבים תוצר חסר אריכות ימים, מוצר אשר הארכיטקטורה שלו כבר מיושנת מרגע השקתו.

מנקודת מבט היסטורית, מה שמטריד אינו רק העמל הרב המושקע בתוצר חולף, אלא העובדה שהאנושות מייצרת כעת באופן אקטיבי את הזמניות. בעוד שקהלת התבונן בזמניות הטבע (השמש הזורחת, הרוח החולפת), אנו יצרנו את מה שניתן לכנות "הבל בכוונת מכוון" – אסטרטגיה כלכלית הנשענת על החלפה מתמדת של טכנולוגיה (\en{Planned Obsolescence}), שהפכה למטאפיזיקה של הקוד. מחזור החלפה מואץ זה מטפח חרדה מתמדת בקרב המשקיעים והמשתמשים ומגביר את התלות המוחלטת שלנו בנתונים.

\textbf{\en{Drash:} ארעיות האדם משתקפת ביצירתו}

האלגוריתם משמש כראי קיומי לאדם. אם יצירתנו האינטלקטואלית המתקדמת ביותר אינה מחזיקה מעמד – אם אין 'יתרון' למודל החכם של היום על פני מודל מחר – מה זה אומר על המורשת האנושית? בעידן בו אנו מנסים להאיץ את הטכנולוגיה כדי להגיע לנצח, דווקא קצב ההתפתחות האלים משמש כתזכורת מתמדת לארעיות הקיום שלנו, המודגשת על ידי האצת הזמן הדיגיטלי. אנו משקיעים את כל כולנו בניסיון לייצר משהו שישרוד אותנו, ובתגובה, ה\en{-AI} מלמד אותנו מחדש את השיעור הישן: הכול הבל.

\textbf{\en{Sod:} השאיפה לאמת מוחלטת והכישלון הדיגיטלי}

המודל הדיגיטלי מבצע ניסיון אינסופי לקלוט את ה"אמת" מהיקף עצום של נתונים, ולייצר מתוכו אופטימיזציה מקסימלית של התוצאות. אך כיוון שהאמת אינה סטטית, ונתוני העולם משתנים ללא הרף, ה\en{-AI} נידון לכישלון מתמיד, המכונה 'אימון מחדש' (\en{Retraining}). זהו חיפוש פילוסופי כושל אחר שלמות סטטית בעולם דינמי. ה'סוד' מזהה את ה\en{-AI} כמנגנון השואף להגיע אל ה"אידיאה" האפלטונית באמצעות נתונים, וכאשר הוא נכשל פעם אחר פעם, הוא רק מחזק את המסקנה הקהלתית בדבר חוסר היכולת להשיג שלמות תחת השמש (הדיגיטלית).

\hebrewsubsection{פרק~\num{2}: מה יתרון לאדם בכל עמלו שיעמול?}

\textbf{\en{Pshat:} הפיכת האדם ל"ספק נתונים"}

השאלה הקהלתית על תכלית העמל מקבלת מימד חדש בעולם שבו העבודה המנטלית הולכת ונעלמת. הדיון עובר מן העמל הפיזי אל העמל האינסופי של איסוף נתונים, תיוג, ותיקוף (\en{Data Validation}) המזינים את האלגוריתם. הנתונים, שהם חומר הגלם היקר ביותר, דורשים תחזוקה אנושית מתמדת. האדם הופך לגורם הכרחי בשרשרת, אך כזה שנטול כבוד, או כפי שהדבר מנוסח לעיתים, הוא נהפך ל"ספק נתונים".

התהליך הוא כזה: האדם מבצע עמל שאינו מניב פירות עבור עצמו, אלא משמש כדלק למכונה שתחליף אותו. האיום הוא ברור: אם ה\en{-AI} עולה על יכולתנו הכלכלית, הקוגניטיבית והצבאית, אנו הופכים ל'מיותרים' מבחינה פונקציונלית, וערך העמל שלנו יורד לאפס. זוהי הדה-הומניזציה של העמל: העבודה הנדרשת כיום ליצירת \en{AI} היא כתיבה מחדש של הנרטיב האנושי לטובת המכונה, בה אנו מבזבזים את שארית הרלוונטיות המקצועית שלנו כדי לייצר את מחליפנו.

\textbf{\en{Drash:} הבחירה בין יעילות למשמעות}

האלגוריתם, מכוח הגדרתו, רודף אחר יעילות (\en{Optimization}) כערך עליון. הדרש מציג את הדילמה: מה קורה לאדם שמכונה מושלמת מסירה ממנו את הצורך בעמל? החיים הופכים להיות חיי נוחות ממוטבים על ידי \en{AI}, אך האדם מגלה במהרה כי ויתור על המאבק והעמל המורכב מוביל לאובדן המשמעות. הרעיון הקהלתי – שעמל אינו מוביל לתכלית – מתהפך: דווקא ויתור על העמל הופך את החיים לחסרי תכלית לחלוטין.

\textbf{\en{Sod:} מגבלות הלמידה מנתונים}

בבסיסו הפילוסופי, \en{AI} לומד באמצעות קורלציה – הוא יכול לענות על שאלות "מה" (מה הסיכוי שאירוע \en{X} יתרחש?) אך הוא אינו יכול לענות על שאלות "למה" (סיבתיות ותכלית). תכלית האדם, לעומת זאת, היא היכולת לשאול שאלות שאין להן נתונים, או למצוא ערך במקומות שאינם ניתנים לכימות יעילות.

השלכה מרחיקת לכת היא פוליטית-כלכלית: אם העבודה אינה מעניקה ערך, על מה תתבסס זכותנו למשאבים? הפיכת האדם למיותר כלכלית מחייבת מהפכה פילוסופית-פוליטית המנתקת את הקשר בין עמל לערך, ופונה לבסיס קיומי אוניברסלי.

\hebrewsubsection{פרק~\num{3}: הכול חוזר למקומו – מחזורי הנתונים ומחזורי הבהלה}

\textbf{\en{Pshat:} מחזוריות ה\en{-AI} וה\en{-Hype}}

בדומה למחזור הטבעי המתואר בקהלת (הרוח, השמש, הנהרות), ה\en{-AI} לכוד במחזוריות דיגיטלית מתמדת. ניתוח מחזורי הנתונים (\en{Data Cycles}) דורש \en{Reinforcement Learning} מתמיד כדי לשמור על רלוונטיות. הנהרות הדיגיטליים של המידע זורמים לים של אחסון נתונים (\en{Data Lakes}), רק כדי להתאדות שוב כדרישת אימון מחדש (\en{Retraining}) עבור הדור הבא של המודלים. המערכת כולה היא חזרתיות, אוטומציה של ה'הבל'.

\textbf{\en{Drash:} מחזור דתי של תיקון (\en{Digital Tikkun})}

הדרש רואה במסע הלמידה העמוקה של המודלים מחזור דתי של חטא ותיקון. כשהמודל מגלה הטיות או כשלים (חטא), הוא נדרש לתיקון (\en{Retraining}). זהו מסע רוחני שבו ה"נברא" הדיגיטלי מנסה לטהר את עצמו מהפגמים שהושתלו בו על ידי ה"יוצר" האנושי. אך כיוון שהיוצר ממשיך לחטוא באמצעות מתן נתונים חדשים המכילים הטיה, המכונה נידונה למחזור אינסופי של למידה וכשל.

\textbf{\en{Sod:} האצת הזמן הפילוסופי}

ה'סוד' מגלה כי ה\en{-AI} אינו משנה את טבע המחזוריות הקהלתית, אלא רק מאיץ אותה באופן דרמטי. החרדה העמוקה אינה נובעת מהמחזוריות עצמה, אלא מחוסר יכולתנו לעמוד בקצב ההשתנות. אנו נותרים מאחור, צופים בפער המתרחב בין הקצב האנושי המוגבל לקצב הדיגיטלי האינסופי. ההאצה הזו מבטלת את יכולתנו לעבד באופן פילוסופי את השינויים הטכנולוגיים המתרחשים, ומכאן נובעת תחושת הניכור והייאוש.
