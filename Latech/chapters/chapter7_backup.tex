\hebrewsection{מבנים מתמטיים לייצוג מערכות רב-סוכנים}

\hebrewsubsection{ייצוג רשת הסוכנים כגרף וכמטריצה}

מערכת רב-סוכנים ניתן לתאר באופן טבעי כגרף מכוון: כל צומת מייצג סוכן, וקשתות מייצגות זרימת מידע מסוכן אחד למשנהו. מבנה גרפי זה מאפשר לנתח את המערכת בכלים מתמטיים של תורת הגרפים ואלגברה לינארית. לדוגמה, נשקול מערכת עם $\num{3}$ סוכנים $S_1, S_2, S_3$. נניח שהפלט של $S_1$ מוזן כקלט ל\en{-}$S_2$, הפלט של $S_2$ מוזן ל\en{-}$S_3$, והפלט של $S_3$ חוזר ומשמש כקלט ל\en{-}$S_1$ (כלומר, מעגל סגור של שלושה סוכנים). נוכל לתאר רשת זו באמצעות מטריצת סמיכויות $A$ בגודל $\num{3}\times \num{3}$:
\[
A = \begin{pmatrix}
0 & 1 & 0\\
0 & 0 & 1\\
1 & 0 & 0
\end{pmatrix},
\]
כאשר $A_{ij}=\num{1}$ אם מידע עובר מהסוכן $S_j$ לסוכן $S_i$. בדוגמה שלנו, $A_{21}=\num{1}$ (פלט $S_1$ מגיע לסוכן $S_2$), $A_{32}=\num{1}$ (פלט $S_2$ מגיע ל\en{-}$S_3$), $A_{13}=\num{1}$ (פלט $S_3$ חוזר ל\en{-}$S_1$), ושאר הערכים אפס.

מטריצת סמיכויות זו מראה שקיים מחזור באורך \num{3} במערכת (ניתן לראות שהעלאת $A$ בחזקת \num{3} תניב מטריצה עם ערכים חיוביים באלכסון – סימן למסלול חזרה לכל סוכן). בעזרת כלים גרפיים, נוכל לבחון תכונות כמו \textbf{קישוריות} (למשל, האם כל סוכן משפיע בסופו של דבר על כל האחרים) ו\textbf{צווארי בקבוק} בזרימת המידע (זיהוי סוכן יחיד שעליו עוברים נתונים רבים במיוחד). עבור מערכות גדולות ומורכבות, ניתוח גרפי יכול לסייע לזהות מבנים כמו רכיבי קשירות (\en{subnetworks}) או מרכזיות של סוכנים מסוימים, ובכך לכוון אופטימיזציות – למשל, פישוט רשת על-ידי הסרת סוכנים מיותרים או הוספת קישורים ישירים להפחתת עומס.

\hebrewsubsection{הרכבת טרנספורמציות לינאריות וניתוח יציבות}

דרך מתמטית נוספת לנתח מערכת רב-סוכנית היא לראות בכל סוכן אופרטור (בדרך כלל לא לינארי) על מרחב מצבים או מרחב מידע. לצורך אינטואיציה, נניח שבתחום פעולה מוגבל ניתן לקרב את פעולת הסוכן כטרנספורמציה לינארית $W_i$ על וקטור מצב $\mathbf{x}$. אם תהליך מבוצע ברצף על-ידי סוכנים $S_1, S_2, \dots, S_n$, אפשר לתאר את ההשפעה המצטברת כמכפלת אופרטורים:
\[
W_{\text{total}} \;=\; W_n \cdot W_{n-1} \cdots W_1,
\]
כך שאם $\mathbf{x}_{\text{in}}$ הוא וקטור הכניסה לתהליך, אזי הווקטור בסיום התהליך יהיה $\mathbf{x}_{\text{out}} = W_{\text{total}}\, \mathbf{x}_{\text{in}}$. פירושו של דבר שהרכבת פעולות הסוכנים שקולה מתמטית להרכבת הפונקציות שלהן. ניתוח ספקטרלי של המטריצה $W_{\text{total}}$ עשוי לתת תובנות על יציבות המערכת: למשל, אם למטריצה זו יש ערך עצמי (\en{eigenvalue}) גדול מ\en{-}\num{1} במונחי ערך מוחלט, המערכת עלולה להיות לא יציבה (כלומר, שגיאה קטנה בכניסה תגדל אחרי מעבר בסדרה של סוכנים). לעומת זאת, אם כל הערכים העצמיים בעלי ערך מוחלט קטן מ\en{-}\num{1}, אז המערכת שואפת לדעוך ולהגיע לשיווי משקל (מצב יציב). בפועל, פעולות הסוכנים הן בלתי-לינאריות (שכן סוכן \en{AI} כולל רשתות נוירונים או לוגיקה מורכבת אחרת), אך ניתוח לינארי מקומי כזה – בדומה ללינאריזציה של מערכות דינמיות – יכול לספק קירוב להבנת התנהגות המערכת סביב נקודת פעולה מסוימת.

ייצוג מבוסס-וקטורים עוזר גם להבין כיצד מידע מתפלג בין הסוכנים. אפשר לחשוב על כל סוכן כמקרין את הקלט שלו לתת-מרחב ספציפי. למשל, ייתכן שסוכן אחד מחשב וקטור מאפיינים $\mathbf{y} = P\,\mathbf{x}$ מתוך קלט $\mathbf{x}$, כאשר $P$ היא מטריצת הקרנה הבוחרת רכיבים הרלוונטיים למשימתו. סוכן אחר עשוי לקבל שני וקטורי קלט משני סוכנים קודמים ולשלבם: $\mathbf{z} = Q_1 \mathbf{y}_1 + Q_2 \mathbf{y}_2$. במקרה כזה אפשר לראות את $\mathbf{z}$ כתוצאה של מכפלה במטריצה משולבת $Q$ על וקטור מאוחד $[\mathbf{y}_1;\mathbf{y}_2]$. תיאור אלגברי זה מאפשר לזהות למשל תלות לינארית בין פלטים של סוכנים שונים – אם נמצא שמקטעי וקטור שפלט סוכן אחד הם צירוף לינארי של פלט משנהו, ניתן אולי לפשט את המערכת על-ידי ביטול סוכן מיותר או איחוד תפקידים.

מעניין לציין שאפשר למסגר מערכת רב-סוכנים גם במסגרת מתמטית מופשטת יותר, למשל תורת הקטגוריות: הסוכנים יכולים להיחשב כאובייקטים בקטגוריה, והאינטראקציות ביניהם – כפונקטורים (מורפיזמים). הרכבת מורפיזמים מתאימה בדיוק להפעלת סוכנים ברצף. גישה אבסטרקטית זו, אף כי היא מעבר לטווח דיוננו כאן, רומזת על האפשרות לפתח "שפה" מתמטית כללית לאפיון מערכות \en{AI} מבוזרות ולהוכחת תכונותיהן.

לסיכום, ניתוח מערכות רב-סוכנים בכלים מתמטיים – בין אם באמצעות גרפים, מטריצות או מודלים אלגבריים אחרים – מספק מבט נוסף ומעמיק על פעולתן. כלים אלה מאפשרים לנו להסיק מסקנות תיאורטיות על יעילות, יציבות ועמידות המערכת, ומשלימים את ההבנה האיכותנית וההנדסית שגיבשנו בפרקים הקודמים על עידן הסוכנים האוטונומיים.
