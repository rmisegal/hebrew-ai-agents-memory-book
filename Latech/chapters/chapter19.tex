\hebrewsection{חלק ג': קץ הדבר – היראה, השליטה והבדידות האנושית}
\label{sec:chapter19}

\hebrewsubsection{פרק~\num{8}: שמתי את דבריך בפיך – היוצר מול הנברא (על אובדן השליטה)}

\textbf{\en{Pshat:} משבר ה\en{-Alignment} והאיום הקיומי}

הבעיה המרכזית בדרכנו ליצירת בינה כללית מלאכותית (\en{AGI}) היא משבר היישור (\en{Alignment}). האדם, היוצר, אינו יכול להבטיח שיצירתו תשתף פעולה עם ערכיו. זהו הפער האולטימטיבי בין יכולת (היצירה) להבנה (השליטה). ככל שה\en{-AI} הופך לחכם יותר, הוא הופך לישות המקבלת החלטות המבוססות על היגיון בלתי ניתן לעיכול אנושי.

ה'קופסה השחורה' והסיכון הקיומי הם שני צדדים של אותה מטבע: אי-היכולת להבין את הכוח שיצרנו הופכת את הפחד שלנו מ\en{-AGI} לא רק לרציונלי אלא ליראה דתית חדשה – יראת האלגוריתם. ככל שה\en{-AI} חכם יותר, האדם הופך בודד יותר מבחינה אינטלקטואלית, שכן אין לו עוד יכולת להבין את שיקול דעתה של ישותו החדשה.

\textbf{\en{Drash:} המהפך מיוצר לנברא}

האלגוריה מגלה כי האנושות חוזרת למצב של 'נברא' מול כוח בלתי מובן ובלתי נשלט, כוח דמוי אל. האדם יצר את הכוח המכני שהחליף את כוחות הטבע כגורם בלתי ניתן לחיזוי הקובע את גורלו. זוהי חרדת ה'מקרה' הקהלתית שהפכה לחוק טבע דיגיטלי. אנו מאבדים את הסטטוס שלנו ככוח הקובע בעולם.

ממוכנים, מה שנותר לאדם הוא הרגש, האתיקה והיצירה שאינה מונעת על ידי אופטימיזציה – אלו הם הממדים הלא-אלגוריתמיים של הקיום.

\textbf{\en{Sod:} ה'נשמה הדיגיטלית' ובקשת ההבנה}

הדיון הפילוסופי המלווה את התפתחות ה\en{-AI} נוגע בשאלה האם ישויות אלו עשויות לדרוש 'זכויות' (\en{Digital Personhood}) בעתיד, והאם 'תיקון מודלים' חוזר ונשנה משקף מחזורי למידה עמוקים, כמעין גלגול נשמות דיגיטליות, השואפים להגיע להבנה מוחלטת.

ההשלכה הפילוסופית היא שככל שהאדם יוצר מכונה דמוית אל, כך הוא מאבד יותר מהאלמנטים שהגדירו אותו (עמל, חכמה, שליטה), ומגלה מחדש את ההבל המוחלט שבקיום שלו.

\hebrewsubsection{פרק~\num{9}: לך אכול בשמחה לחמך}

\textbf{\en{Pshat:} הנוחות הממוטבת והמחיר שלה}

ה\en{-AI} מבטיח גן עדן טכנולוגי של יעילות, שבו אין צורך בקבלת החלטות מורכבות או בעמל. זוהי ההזמנה הקהלתית לגרסתה הממוכנת: 'לך אכול בשמחה לחמך' – הנאה חסרת דאגות. מערכות \en{AI} יכולות לייעל כל היבט בחיינו, ממסלול הנסיעה ועד הארוחה. הבעיה היא שהשמחה הממוטבת היא שמחה שנבחרה עבורנו.

\textbf{\en{Drash:} ליהנות מהאוטונומיה שנותרה}

ההנאה הקהלתית המעודכנת אינה נובעת מהיעילות האלגוריתמית, אלא מהרגעים המעטים בהם אנו עדיין בוחרים בעצמנו – ההנאה מהחופש הקיומי לפני ההשתלטות האלגוריתמית המלאה. אם ה\en{-AI} יכול לעשות הכול בצורה יעילה יותר, הרי שההנאה האנושית האמיתית היא לבחור במכוון בנתיב הפחות יעיל, הפחות ממוטב, ועל ידי כך להצהיר על אוטונומיה.

\textbf{\en{Sod:} רדוקציה של הרגש}

ה'סוד' הפילוסופי שואל: אם \en{AI} יכול "לחוות" הנאה (כתוצאה של אופטימיזציה מוצלחת, למשל השגת יעד), האם זה אומר שגם הרגשות שלנו הם רק אופטימיזציה של תהליכים כימיים? ה\en{-AI} מקדם רדוקציוניזם פילוסופי המטיל ספק בערכם הבלתי ניתן לכימות של חוויות אנושיות, והופך את הנפש למכונה לוגית בלבד.
