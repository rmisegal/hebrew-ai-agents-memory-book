\hebrewsection{מקהלת הסוכנים: שילוב עם \en{Claude CLI}}

סוכן בודד – חזק ומועיל ככל שיהיה – מגיע למלוא עוצמתו רק כשהוא חלק מתזמורת של סוכנים. לאחר שבפרק \num{3} בנינו סוכן \en{MCP} מלא עבור \en{Gmail}, כעת מטרתנו היא לשלב אותו בתוך מנגנון אורקסטרציה רחב יותר באמצעות \en{Claude CLI}. פלטפורמה זו משמשת כ"מנצח" המנהל מספר סוכנים מומחים במקביל, בהתבסס על פרוטוקול \en{MCP}. שילוב הסוכן מאפשר להפעילו באמצעות שפה טבעית כחלק מהאינטראקציה עם \textbf{Claude}, ובכך לשרשר תת-משימות באופן אוטומטי וחלק.

להשלמת האינטגרציה, עלינו לבצע מספר צעדים טכניים:
\begin{enumerate}
  \item \textbf{הגדרת שרת ב\en{-Claude CLI}:} נערוך את קובץ התצורה של \en{Claude CLI} כדי לרשום את שרת ה\en{-MCP} שלנו. למשל, נוסיף במקטע \en{\texttt{mcpServers}} כניסה עבור "\en{\texttt{gmail-extractor}}" המצביעה על הפקודה להפעלת שרת הסוכן (ראו דוגמה בנספח ג).
  \item \textbf{רישום יכולות הסוכן:} ניצור קובץ תיאור לסוכן (למשל \en{\texttt{gmail-extractor.md}}) המפרט את תפקידו, יכולותיו, שם השרת (\en{\texttt{gmail-extractor}}) ופרטי הכלי שהוא מספק (ראו נספח ג למלל המלא).
  \item \textbf{אימות וחיבור:} נפעיל את \en{Claude CLI} ונוודא שהסוכן החדש נטען בהצלחה ברשימת הסוכנים הזמינים (\en{\texttt{claude agents list}} יציג את \en{\texttt{gmail-extractor}}). לאחר מכן נוכל לנסות פקודת בדיקה בשפה טבעית, למשל: \en{\texttt{/agent use gmail-extractor to fetch emails with the label "INBOX" from the last 7 days}}. כעת נצפה ש\en{-Claude} יאתחל את הסוכן, יבצע אימות \en{OAuth} (בפעם הראשונה), יריץ את החיפוש, ולבסוף יחזיר תגובת \en{JSON} המכילה את התוצאות (לדוגמה: \en{\{\texttt{"success": true, "count": 12, ...}\}}).
\end{enumerate}

לאחר השלמת שלבים אלו, הסוכן שלנו משולב באופן מלא במערכת. כעת משתמש קצה יכול לבקש מ\en{-Claude}, כחלק משיחה רגילה, לבצע פעולות המבוססות על הסוכן (כגון "חפש עבורי אימיילים עם תווית X מהחודש האחרון"), ו\en{-Claude} יפנה את הבקשה אל הסוכן המתאים, ימתין לתוצאתו, ואז יסכם למשתמש את המידע שהתקבל.

חשוב להדגיש ש\textbf{Claude CLI} תומך בהפעלת סוכנים מרובים בו-זמנית. המשמעות היא שנוכל להוסיף למערכת סוכנים מתמחים נוספים (למשל, סוכן לניתוח נתונים או סוכן לחילוץ מידע מרשתות חברתיות) ולתזמר ביניהם. החיבור דרך פרוטוקול \en{MCP} מאפשר לכל סוכן לפעול בבידוד עם הקשר וכלים משלו, בעוד \en{Claude} משמש כליבה מרכזית המתזמרת את שיתוף הפעולה ביניהם\cite{Anthropic2025}. באופן זה ניתן לבנות "מקהלה" של סוכני \textbf{AI} העובדים בהרמוניה להשגת מטרות מורכבות במיוחד.
