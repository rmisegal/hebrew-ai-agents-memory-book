\hebrewsection{ארבעת עמודי הזיכרון המובנה}

\hebrewsubsection{מעבר מהארכיטקטורה ליישום: ארבעת הקבצים}

לאחר שהבנו את ההבחנות הארכיטקטוניות בין פתרונות הזיכרון השונים, אנו מגיעים לליבת החידוש המעשי: מערכת ארבעת הקבצים של \en{Claude Code Memory}. מערכת זו, שהתגבשה מתוך ניסיון מעשי של אלפי פרויקטים, מייצגת פתרון הנדסי אלגנטי לבעיית הזיכרון הפרסיסטנטי.

כפי שראינו בפרק \num{4}, \en{Claude CLI} מספק את התשתית לאורכסטרציה של סוכנים מרובים. אולם בלי מנגנון זיכרון, כל הפעלה של הסוכן היא "נקודתית" – היא מתחילה מאפס ומסתיימת ללא זכר. ארבעת הקבצים הם הפתרון למעבר מסוכן רגעי לשותף קוגניטיבי.

\hebrewsubsection{עמוד ראשון: \en{\texttt{PRD.md}} – מסמך דרישות המוצר}

\textbf{\en{Product Requirements Document (PRD)}} הוא הקובץ הראשון והאסטרטגי ביותר. הוא משיב על השאלה הבסיסית: \textbf{מה אנחנו בונים?} תפקידו אינו טכני אלא עסקי-אסטרטגי: להגדיר את החזון, את היעדים, ואת קריטריוני ההצלחה של הפרויקט.

מבנה אופייני של \en{\texttt{PRD.md}} כולל:
\begin{itemize}
  \item \textbf{חזון אסטרטגי} (\en{Vision}): מדוע הפרויקט קיים? מה הוא מנסה להשיג?
  \item \textbf{יעדים ומטריקות} (\en{Objectives and Metrics}): כיצד נמדוד הצלחה? (למשל, \num{50}–\num{58} עמודים, \num{100}\% תאימות \en{CLS})
  \item \textbf{דרישות פונקציונליות} (\en{Functional Requirements}): מה המערכת צריכה לעשות?
  \item \textbf{דרישות לא-פונקציונליות} (\en{Non-Functional Requirements}): איכות, ביצועים, אבטחה
  \item \textbf{קריטריוני קבלה} (\en{Acceptance Criteria}): מתי נחשיב את הפרויקט כ"הושלם"?
\end{itemize}

בפרויקט זה, למשל, ה\en{-PRD} הגדיר את ההרחבה מגרסה \num{3.0} (חלק אחד, \num{6} פרקים) לגרסה \num{4.0} (שני חלקים, \num{13} פרקים), עם דגש על שמירת רמת הנגישות של הרארי ועל הקפדה מוחלטת על \num{100}\% תאימות \en{CLS}.

\hebrewsubsection{עמוד שני: \en{\texttt{CLAUDE.md}} – ספר החוקים הקנוני}

\textbf{\en{CLAUDE.md}} הוא הקובץ המחייב והמאכף ביותר. הוא משיב על השאלה: \textbf{איך אנחנו עובדים?} מדובר ב"חוקת הפרויקט" – מערכת כללים קשיחה שאינה ניתנת למשא ומתן.

תפקיד ה\en{-CLAUDE.md} הוא כפול:
\begin{enumerate}
  \item \textbf{אכיפת מגבלות טכניות}: למשל, "השתמש אך ורק ב\en{-LuaLaTeX}", "אל תשתמש ב\en{\textbackslash textenglish} או \en{\textbackslash texthebrew}", "השתמש ב\en{\textbackslash en\{\}} לכל טקסט אנגלי".
  \item \textbf{הנחיית תהליך עבודה}: למשל, "קרא את \en{\texttt{PLANNING.md}} בתחילת כל סשן", "סמן משימות כהושלמו מיד לאחר השלמתן", "בצע קומפילציה לאחר כל שינוי".
\end{enumerate}

הקובץ כולל גם סטנדרטים איכותיים – למשל, "כל פרק בחלק \num{2} חייב לעמוד בסטנדרט הרארי: \num{80}\%+ נגישות למי שאינו מומחה, פתיחה בהקשר היסטורי, הגדרת מונחים מיד עם השימוש הראשון".

ה\en{-CLAUDE.md} הוא הכלי הקוגניטיבי החזק ביותר: הוא \textbf{מכריח} את הסוכן לפעול בצורה ממושמעת, ובכך מונע סחף קונטקסטואלי וטעויות חוזרות.

\hebrewsubsection{עמוד שלישי: \en{\texttt{PLANNING.md}} – האסטרטגיה הטכנית}

\textbf{\en{PLANNING.md}} הוא מסמך הארכיטקטורה הטכנית. הוא משיב על השאלה: \textbf{איך נגיע ליעד?} אם ה\en{-PRD} הוא "מה", וה\en{-CLAUDE.md} הוא "איך נעבוד", הרי ה\en{-PLANNING.md} הוא "איך נבנה".

תוכן אופייני של \en{\texttt{PLANNING.md}} כולל:
\begin{itemize}
  \item \textbf{מבנה טכנולוגי}: רשימת טכנולוגיות, ספריות, כלים (\en{LuaLaTeX, BibLaTeX, Polyglossia, Bidi})
  \item \textbf{מבנה קבצים}: מיפוי מדויק של הספריות והקבצים (למשל, \en{\texttt{chapters/chapter1.tex}} עד \en{\texttt{chapter13.tex}}, \en{\texttt{claude\_mem\_part2/}})
  \item \textbf{מיפוי פרקים}: התאמה בין קטעי קוד מקור (למשל, \en{PDF Section 4}) לבין פרקים ב\en{-LaTeX} (\en{chapter10.tex})
  \item \textbf{אסטרטגיית הפניות צולבות}: כללים ליצירת הפניות קדימה ואחורה בין פרקים
  \item \textbf{פירוק לשלבים} (\en{Phase Breakdown}): למשל, \num{10} שלבים מתכנון ועד תיעוד סופי
\end{itemize}

בפרויקט זה, ה\en{-PLANNING.md} פירט את המעבר המתוכנן: שלב \num{0} (תכנון), שלב \num{1} (ביבליוגרפיה), שלב \num{2} (המרת טבלה), שלבים \num{3}–\num{7} (המרת פרקים), שלב \num{8} (בדיקות אינטגרציה), שלב \num{9} (סקירת איכות), שלב \num{10} (תיעוד). מסמך זה משמש כ"מפת דרכים" שהסוכן קורא בתחילת כל סשן כדי להבין היכן הוא נמצא במסע.

\hebrewsubsection{עמוד רביעי: \en{\texttt{TASKS.md}} – רשימת המשימות החיה}

\textbf{\en{TASKS.md}} הוא הקובץ הדינמי והמתעדכן ביותר. הוא משיב על השאלה: \textbf{מה עשינו, מה נותר לעשות?} מדובר ב"פנקס הביצוע" – רשימה חיה של כל המשימות שבוצעו ושטרם בוצעו.

מבנה אופייני כולל:
\begin{itemize}
  \item \textbf{תבנית \en{Checkbox}}: \texttt{- [ ] משימה לא הושלמה} לעומת \texttt{- [x] משימה הושלמה (\checkmark{} 2025-10-20)}
  \item \textbf{\en{Milestones} בתוך שלבים}: למשל, "שלב \num{5}: המרת פרק \num{9}" מפורק ל\num{-9} תת-משימות
  \item \textbf{עקרון "סמן מיד"} (\en{Mark Immediately}): סימון משימה כהושלמה \textbf{ברגע ההשלמה}, לא בהמשך
  \item \textbf{הוספת משימות חדשות בזמן אמת}: אם מתגלה צורך בלתי צפוי, הוא מתווסף מיד ל\en{-TASKS.md}
\end{itemize}

ה\en{-TASKS.md} הופך את הפרויקט ל"מפה חיה" שמעודכנת בזמן אמת. זה מונע את תופעת ה"אמנזיה בין-סשנית": סוכן חדש שנכנס לפרויקט יכול לקרוא את \en{\texttt{TASKS.md}}, לראות בדיוק מה הושלם ומה נותר, ולהמשיך בלי להתחיל מחדש.

בפרויקט זה, למשל, \en{\texttt{TASKS.md}} מכיל למעלה מ\num{-150} משימות מפורטות, מסומנות במדויק עם תאריכי השלמה. זה מאפשר מעקב מלא אחר התקדמות הפרויקט.

\hebrewsubsection{המנגנון הקוגניטיבי: תקציב \en{Tokens} ואכיפה}

כדי שמערכת ארבעת הקבצים תעבוד, על ה\en{-LLM} \textbf{לקרוא אותם בתחילת כל סשן}. זהו המנגנון האכיפה הקוגניטיבית: הסוכן "נאלץ" להזריק את התוכן של הקבצים לחלון ההקשר שלו, ובכך הוא "זוכר" את הכללים, המבנה, והמשימות.

הקצאת תקציב \en{Tokens} אופיינית:
\begin{itemize}
  \item \en{\texttt{PRD.md}}: \num{15}–\num{20}\% מחלון ההקשר (קונטקסט אסטרטגי)
  \item \en{\texttt{CLAUDE.md}}: \num{25}–\num{30}\% (אכיפת כללים – הקריטי ביותר)
  \item \en{\texttt{PLANNING.md}}: \num{20}–\num{25}\% (ארכיטקטורה טכנית)
  \item \en{\texttt{TASKS.md}}: \num{20}–\num{25}\% (מצב ביצוע)
  \item שאר ההקשר (\num{10}–\num{20}\%): לקריאות קבצי קוד, תוצאות כלים, דיאלוג עם המשתמש
\end{itemize}

\textbf{תהליך עבודה חובה בתחילת כל סשן:}
\begin{enumerate}
  \item קרא את \en{\texttt{PLANNING.md}} \textbf{קודם כול} – הבן את הארכיטקטורה ואת השלבים
  \item בדוק את \en{\texttt{TASKS.md}} – ראה מה הושלם, מה הבא בתור, מה התלויות
  \item עבוד על המשימה הבאה בתור
  \item סמן משימות כהושלמו \textbf{מיד} עם תאריך (\en{Mark Immediately})
  \item הוסף משימות חדשות שהתגלו תוך כדי עבודה (תפקידו של \en{\texttt{TASKS.md}} להיות "מסמך חי")
\end{enumerate}

מנגנון זה יוצר "לולאת משוב קוגניטיבית": הסוכן קורא → מבין → מבצע → מעדכן → הסשן הבא קורא את העדכון → ממשיך מהנקודה המדויקת שבה הסשן הקודם הפסיק. זו למעשה הדמיה של זיכרון ארוך-טווח.

\hebrewsubsection{מפרויקט חד-פעמי לשותפות ארוכת-טווח}

ההבדל המהותי בין עבודה \textbf{ללא} מערכת הזיכרון לבין עבודה \textbf{עם} המערכת הוא דרמטי:

\textbf{ללא זיכרון} (מודל סטנדרטי):
\begin{itemize}
  \item כל סשן מתחיל מאפס
  \item המשתמש מסביר מחדש "מה הפרויקט, איזה כללים, איך עובדים"
  \item טעויות חוזרות (למשל, שימוש ב\en{\textbackslash textenglish} שוב ושוב)
  \item חוסר עקביות ארכיטקטונית (כל סשן "מחליט" אחרת)
  \item תלות מוחלטת בזיכרון האנושי ("האם עשינו את X? האם תיקנו את Y?")
\end{itemize}

\textbf{עם זיכרון} (מערכת \num{4} הקבצים):
\begin{itemize}
  \item כל סשן מתחיל מהמצב המדויק של הסשן הקודם
  \item הסוכן "יודע" מה הפרויקט, מה הכללים, מה הושלם
  \item אכיפה אוטומטית של כללים (אם \en{\texttt{CLAUDE.md}} אומר "אל תשתמש ב\en{-X}", הסוכן לא ישתמש)
  \item עקביות ארכיטקטונית מלאה לאורך זמן
  \item רציפות ביצועית: התקדמות מצטברת, לא התחלה חוזרת
\end{itemize}

בפרק \num{11} נעמיק בעקרונות המעשיים לניהול ידע בפרויקטים ארוכי-טווח, ונראה כיצד מערכת ארבעת הקבצים משמשת תשתית לשיתוף פעולה אנושי-מכונה מתמשך ופרודוקטיבי. המעבר מ"סוכן כלי" ל"סוכן שותף" מתחיל כאן, בהנדסה פשוטה אך מהפכנית של זיכרון חיצוני מובנה.
