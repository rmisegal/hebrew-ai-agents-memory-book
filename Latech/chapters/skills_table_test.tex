\documentclass{../hebrew-academic-template}
\begin{document}

\hebrewsection{בדיקת טבלאות \en{Skills}}

% ========================================
% Table 1: Skills Comparison Table
% Source: SkillsExplanations.pdf, page 3
% 5 columns × 4 rows
% ========================================

\begin{hebrewtable}[H]
\caption{מודלים להפצת ידע ומומחיות בסביבות \en{AI}}
\label{tab:skills_comparison}
\centering
\begin{rtltabular}{|m{2.8cm}|m{2.8cm}|m{2.8cm}|m{2.8cm}|m{2.8cm}|}
\hline
\hebheader{מאפיין} &
\enheader{Claude Skills (CLI)} &
\enheader{Claude Projects (Web/CLI)} &
\enheader{Custom GPTs (OpenAI)} &
\enheader{Model Context Protocol (MCP)} \\
\hline

\hebcell{מטרת העל} &
\hebcell{אריזת מומחיות פרוצדורלית וקוד ניתנים לשימוש חוזר.} &
\hebcell{ניהול קונטקסט נרחב, מסמכי יסוד (\en{Artifacts}) ו\en{-Data Room}.} &
\hebcell{משימות נישתיות ואינטראקציה מבוססת \en{API}.} &
\hebcell{רישום פורמלי של כלים חיצוניים (\en{APIs}).} \\
\hline

\hebcell{בסיס ארכיטקטוני} &
\hebcell{מערכת קבצים מודולרית (\en{\texttt{SKILL.md}} + \en{Progressive Disclosure}).} &
\hebcell{קונטקסט ווקטורי גדול + הגדרות \en{YAML}.} &
\hebcell{הוראות יסוד (\en{Instructions}) וקובצי ידע.} &
\hebcell{קבצי \en{JSON/YAML} המגדירים סכמות \en{API}.} \\
\hline

\hebcell{עלות \en{Context}} &
\hebcell{נמוכה (טעינת \en{Metadata} בלבד בהתחלה).} &
\hebcell{גבוהה (עלולה לרוקן מכסה עקב טעינת מסמכים חוזרת).} &
\hebcell{נמוכה עד בינונית (מוגבלת בגודל).} &
\hebcell{גבוהה (טעינת הרישום כולו כקונטקסט).} \\
\hline

\hebcell{ניידות/שיתוף} &
\hebcell{גבוהה (תיקיית קבצים, פורמט דה-פקטו ניטרלי).} &
\hebcell{בינונית (משותף בתוך הארגון/צוות).} &
\hebcell{גבוהה (דרך ה\en{-GPT Store}).} &
\hebcell{בינונית (דורש שרת \en{MCP} פעיל).} \\
\hline
\end{rtltabular}
\end{hebrewtable}

\vspace{2em}

% ========================================
% Table 2: Skills Paths Table
% Source: SkillsExplanations.pdf, page 4
% 4 columns × 2 rows (+ header)
% ========================================

\begin{hebrewtable}[H]
\caption{מיקומי תיקיות \en{Skills} ב\en{-Claude CLI} (הקשר המערכתי)}
\label{tab:skills_paths}
\centering
\begin{rtltabular}{|m{2.5cm}|m{4cm}|m{4cm}|m{3cm}|}
\hline
\hebheader{סוג \en{Skill}} &
\enheader{נתיב בתוך Linux (כולל WSL)} &
\enheader{נתיב משוער ב-Windows (בהקשר של WSL)} &
\hebheader{משמעות ארכיטקטונית} \\
\hline

\encell{Personal Skill} &
\encell{\textasciitilde/.claude/skills/} &
\encell{/home/<user>/.claude/skills/ (בתוך סביבת WSL)} &
\hebcell{זמינות גלובלית; מומחיות אישית וניסיונית.} \\
\hline

\encell{Project Skill} &
\encell{./.claude/skills/ (בתוך ה\en{-Repo})} &
\encell{./.claude/skills/ (בתוך ספריית הפרויקט הממופה)} &
\hebcell{עקביות צוותית; נכנס ל\en{-Git}.} \\
\hline
\end{rtltabular}
\end{hebrewtable}

\vspace{2em}

\textbf{הערות}:
\begin{itemize}
  \item טבלה~\ref{tab:skills_comparison} משווה בין \num{4} פלטפורמות להפצת מומחיות: \en{Skills}, \en{Projects}, \en{Custom GPTs}, ו\en{-MCP}.
  \item טבלה~\ref{tab:skills_paths} ממפה את מיקומי הקבצים עבור \en{Skills} אישיים ופרויקטליים.
  \item שתי הטבלאות משתמשות ב\en{-RTL} עברי עם תאים מעורבים (\texttt{\textbackslash hebcell\{\}} ו\texttt{\textbackslash encell\{\}}).
\end{itemize}

\end{document}
